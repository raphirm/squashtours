% !TeX spellcheck = de_CH
\chapter{Implementation}

Die Applikation ist unter der URL https://github.com/raphirm/squashtours abrufbar.

\section{Environment}
Das Setup benutzt folgende Komponenten:
\begin{itemize}
\item Spring Framework 4 (last release): Web Framework welches einzelne Webseiten und eine API bereitstellt
\item Spring Security 3 (last release): Sicherung des Web Frameworks
\item Spring Boot 1 (last release): Minimal-Configuration Spring und Spring Security Setup. Bietet einen einfacheren einstieg in Spring und die relevanten Komponenten k�nnen im nachhinein per Java konfiguriert werden.
\item MySQL 5.3: Datenbank um Nutzer und Rollen abzuspeichern

\end{itemize}

Das Ziel dieser Implementation ist, Spring Security zu implementieren und einige Features zu benutzen. Folgende Features wurden in der Implementation benutzt:
\begin{itemize}
\item Multi-Method authentication: Die Webseite benutzt ein Login-Formular, w�hrend die API Requests �ber Basic-Authentication authentisiert
\item Eigene implementationen von UserDetail und UserDetailService: UserDetail wurde implementiert und interagiert direkt mit der MySQL datenbank.
\item Eigene Login-, Logout- und Access Denied Seiten f�r die Webseite.
\item Bcrypt Passwort Encryption: State-of-the-art Passwort Encryption f�r eine sichere Sicherung der Passw�rter in der Datenbank.
\item CSFR Protection f�r Webseite, jedoch nicht f�r API. 

\end{itemize}

\section{Details zur Implementation}
