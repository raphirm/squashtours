% !TeX spellcheck = de_CH


\chapter{Inhalt der Arbeit}
Spring ist ein weit verbreitetes J2EE Framework, das vor allem im komplexeren Umfeld eingesetzt wird. Viele grosse Unternehmen setzen auf das Framework. Um eine sichere Website zu erstellen, gibt es ein Add-On namens Spring Security, welches viele Funktionen im Bereich Sicherheit, Authentisierung und Autorisierung mitbringt. Dieses Add-On wird in dieser Arbeit untersucht, erkl�rt sowie implementiert in einer konkreten Applikation. Verschiedene Use-Cases werden dabei getestet und mit einem Penetrationtest auf die Wirksamkeit gepr�ft.

\section{Motivation}
Als Sicherheitsexperte ist der Autor dieses Dokumentes sehr interessiert in standardisierten Sicherheitsframeworks und ob diese auch wirklich halten was sie versprechen. Webseiten sind im heutigem Umfeld das meist attackierte Ziel und m�ssen darum entsprechend gesch�tzt werden. Eine gehackte Website richtet nicht nur einen Imageschaden oder einen Ausfall an. Sie dient auch als Malware-Verbreiter und kann als Einfallstor direkt in die Firma dienen und zum Datenklau f�hren. 

\section{Abgrenzung und Inhalt}
Diese Arbeit besch�ftigt sich mit der Konfiguration, Konzepten und praktischer Implementation von Spring Security. Das Spring Framework wird angesprochen, Erkl�rungen werden jedoch nur geliefert wenn nicht anders m�glich. 

\section{Vorgehen}
Der Autor wird sich im Detail mit dem Spring Framework, besonderem Spring Security auseinandersetzen und die Einzelnen Features und Konzepte verstehen und beschreiben. Anschliessen wird eine Implementation von Spring Security erstellt. 


\section{Zielsetzung der Arbeit}
Die Arbeit besteht aus zwei Komponenten. Der erste Teil setzt sich mit dem Konzept von Spring Security auseinander und zeigt auf welche Komponenten es gibt und wie diese miteinander Interagieren.

Als zweiter Teil wird eine Web Applikation praktisch gesichert mit Spring Security. Folgende Richtlinien sollten dabei erf�llt werden: 

\begin{itemize}
\item Eine sichere Implementation der Applikation mit dem Spring Framework.
\item Es sollten m�glichst viele Spring-Built-in Sicherheitsfeatures verwendet werden.
\item Eine Dokumenation die Begr�ndet warum welche Sicherheitsfeatures verwendet wurden und wie die verwendeten Sicherheitsfeatures funktionieren.
\item Einen Penetration Test der ansatzweise "Beweisen" sollte, dass das System relativ sicher ist.
\end{itemize}
 
