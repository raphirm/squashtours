\chapter{Reflexion}
\section{Planung und Tests}
Da ich nicht angenommen habe, dass eine Android App zu programmieren so komplex ist. Und zus�tzlich die API und die Android App das erste Projekt dieser beiden Klassen waren, habe ich einige Fehleinsch�tzungen gemacht. 

Nichts desto trotz habe ich die Minimalziele, welche in der Aufgabenstellung beschrieben wurden erreicht. Durch den Zeitmangel habe ich grosse Abstriche im Testing hinnehmen m�ssen, und habe die Appikationen nur von Hand getestet. Ich sch�tze die Stabilit�t der Applikation als Alpha Version ein.

\section{Funktionalit�t}
Ich habe viel zu sp�t bemerkt, dass ich viel zu viel Funktionalit�t umsetzen wollte, welche ich aus Zeitgr�nden gar nicht konnte:
\begin{itemize}
\item Ein HTML5 Webinterface neben der API
\item Bessere Sicherheit und User-Based Filtering
\item Ort des Matches ber�cksichtigen
\item Automatische Herausforderungen der Liga (z.b. alle 2 Wochen )
\item Remove-Endpoints f�r alle Ressourcen
\end{itemize}

Diese Funktionalit�ten wurden darum Descoped und w�rden in einem n�chsten Release der Applikation (inklusive Tests) umgesetzt werden. Die API sowie die Applikation w�rde eine Weiterentwicklung problemlos unterst�tzen. Ich habe zus�tzlich darauf geachtet das die Applikation in Zukunft gut erweitert werden kann.

\section{Hibernate-Bug und Workaround}
W�rend der Arbeit trat ein Hibernate Bug auf. Die Funktion findOne(id) gab ein falsches Resultat zur�ck. Ich konnte ein Workaround herstellen, indem ich findAll() aufrufte und mein Objekt aus dem Array aus Objekten herauslas.

